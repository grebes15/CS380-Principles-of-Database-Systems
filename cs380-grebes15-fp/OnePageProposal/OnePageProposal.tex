\documentclass{article}
\begin{document}
\noindent
Andreas Landgrebe
\\
November 19, 2014
\\
Computer Science 380
\\
\textbf{5. XML Applications with DOM and SAX}
Learn how to write Java programs with simple API for XML (SAX) or document object model (DOM) parsers and then implement and test at least one cast study application that use these different XML processing techniques. Students who pick this project should study how the use of different XML parsers influences your implementation profess and the overall performance of your applications.
\section{Project Assignment and Project Proposal}
\textbf{Comparing and Contrasting DOM and SAX: Picking the best parsing method}
\par
\indent
In this project, I will be comparing the two most popular parsing method, DOM and SAX. In this project, I will implement the two methods and compare and contrasts these parsing methods. Besides, implemented the two methods, I will compare the time metrics between the two and see the performance between the two whether which one is quicker. I will also be testing the two methods with different XML files. I will get these XML from the world wide web and test different sizes whether the XML file is large or small, I will testing a wide range of sizes of XML files. After the implementation, I will also research the specific features that the two have. In this section, I will compared and contrast the two method and see which features in one of the parsing methods and the features that are not available in the other parsing method. After I explain the comparison and contraction between the two parsing methods, the next thing I will do for this project is to explain the implementation between the two methods. In this section, I will explain the whether it was easier to implement the DOM or the SAX parsing method. I will explain the complexity of the implementation will discuss any future implementation whether it would be easier to implement the parser in DOM and SAX. The last section that I will write is explaining the output of the two. In this section I will explain what output was more human readable as a default and explain what the output displays between the DOM and SAX parsing method.
\par
\indent
Using the past laboratory assignment is a great starting point for this final project I will be completely. In the past laboratory assignment, we were researching the DOM parsing method approach for XML files. After I look at this starting point, I will be researching about the implementation of the SAX parsing method. After I research after the SAX parsing method, the next thing to do is look at different XML files to perform my tests with. It will be important to use a wide range of sizes for the XML files to be able to compare the two parsing methods and see if there is a big difference in performance between the two parsing methods.
\end{document}