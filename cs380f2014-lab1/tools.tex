\documentclass{article}
\usepackage{hyperref}
\begin{document}
\noindent
Andreas Landgrebe
\\
Laboratory Assignment \# 1 : Introducing Data Management Concepts and Tools
\\
\\
\textbf{4. A comprehensive listing of ten free and open-source data management tools}
\begin{enumerate}
\item MySQL
\\
\url{http://www.mysql.com/}
\\
MySQl is the world's second most widely used open source relational database management system. It is currently owned by Oracle Corporation. It is a major component for the open source web application software stack LAMP (Linux, Apache, MySQL, PHP). Some of the features include cross-platform support so it is able to support many formats. Another features also includes transactions, embedded database library, multiple storage engines and many more. All of the features that were listed in the past one page write up are features available in MySQL so this data management tool would be a great choice to use.     
\\
\\
\item SQLite
\\
\url{http://www.sqlite.org/}
\\
SQLite is a relational database management system. This system is contained in the C programming language. SQLite is a popular choice to use in embedded database for local/client storage in application software. One of the features that SQLite has is a standalone program called sqlite3 where one can define tables within it, insert, and change rows, and run queries. This standalone program contains many of the features that was written down in the features file. There are many other standalone programs in this data management tool that allows one to have all of the features listed so SQLite is another good candidate to be used for a data management tool. 
\\
\\
\item PostgreSQL
\\
\url{http://www.postgresql.org/}
\\
PostgreSQl is an object-relational database management system. This system is written in C. This system has put an emphasis on extensibility. One of the primary features of this system is top store data, securely and supporting best practices, and retrieve it later. Some other features that this system has is handling workloads ranging from small single-machine application to large Internet-facing application with many concurrent users. This is yet another good candidate to be used for a data management tool. It is able to store data and no matter the size of the system it needs to handle, Postgres is able to be efficient to handle a certain size of a system. 
\\
\\
\item CSQL
\\
\url{http://csql.sourceforge.net/}
\\
CSQL is an open source main memory high-performance relational database management system developed at sourceforge.net. This system is designed to provide high performance for certain storage. Some of the features that this system has having durability to be able to recover all the committed transaction in case of application crash. Another features it has is protection from process failures by freeing resources held by dead application processes. This system also has a direct access to databases so data is available for the client or server. This data management tool has many of the features that were listed before including backup of data, accessing data, supporting concurrence. This is yet another good candidate to use as a data management tool. 
\\
\\
\item TxtSQL
\\
\url{http://sourceforge.net/projects/txtsql/}
\\
TxtSQL is an object-oriented flat file database management system. This system is written all in PHP. Some of the features that this system has is that it is quycker than most other flatfile database scripts which is written PHP. Another feature for TxtSQL is that it is function without any extra software or outside plugins to be downloaded and installed. One of the big features for txtSQL is accessibility and simplicity. Since one of the most important features listed before was simplicity, txtSQL could be a good candidate for this system to be used.  
\\
\\
\item MongoDB
\\
\url{http://www.mongodb.org/}
\\
MongoDB is a cross platform document oriented database. This database is written in C++. One of the features of MongoDB is that it eschews traditional table-based relational database which makes the integration of data in certain types of application easier and faster. Another one of the main features is its file storage so it can be used as a file system which takes advantage of load balancing and data replication over multiple machines for storing files. One of the big features in this database is its accessibility to specific data to be able to store data and retrieve data in a reliable which covers several of the features listed before.   
\\
\\
\item CUBRID
\\
\url{http://www.cubrid.org/}
\\
CUBRID is an open source SQL-based relational database management system with object extensions. It is written in C. Some of the metrics that this database features are high availability, scalability, maintainability, performance, and storage. Some of the metrics that were listed covers all of the features listed before in the context of being able to delete, query, and insert data. It is also scalable so it is able to support many formats. This is yet another good another data management tool that could be used from all of the features listed before.
\\
\\
\item GNOME-DB
\\
\url{http://www.gnome-db.org/}
\\
GNOME-DB is a database application. It is able to access Libgda. Libgda is a databse and a data abstraction layer. It is a library that implements the interfaces for both the client and server parts. It is also a small database access library that has graphical tools depending on GTK+ and GraphViz. This tool is able to have features such as accessing data. This tool has all of the features listed before to be able to maintain, simplify, and create data. 
\\
\\
\item SmallSQL
\\
\url{http://www.smallsql.de/}
\\
SmallSQL is a 100\% pure Java Database Management System for desktop applications. It has no network interface and user management and this is no installation required. Since there is no user management then the user would not be able to have reliability or maintainability. This may have the simplicity, however when to comes to modify data, this database tool would not have this. The reason for this is because this database is meant to target Java desktop applications.
\\
\\
\item HSQLDB
\\
\url{http://hsqldb.org/}
\\
HSQLDB is a relational database management system written in Java. It is a cross platform tool so it is able to access different formats. One of the features of this data management tool is updatable views so it is able to modify and maintain data which is some of the features that were listed before. This could be a good candidate to use as a data management tool since it is reliable and accessible using Java.  

\end{enumerate}
\end{document}