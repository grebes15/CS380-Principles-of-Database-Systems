\documentclass{article}
\begin{document}
\noindent
Andreas Landgrebe
\\
Laboratory Assignment \# 2: Procedural Programming and File Processing System
\\
4. At least one supporting visualization of some trend that you found in the data set
\\.
One supporting visualization of some trend that I found in the data set was the amount of the season that was going on when evaluating these rivers. These visualization was important to look into from the size, and speed of the rivers considering the season that it was in. The interesting part of these visualization is that there will be moments where it will be winter and few algae will arise and then there will be another winter that several algae arise. It There are several contributing factors to determine which algae will arise in specific rivers on whether it is large, medium or small or if it is a slow, medium, or fast river stream. In the computer science classes at Allegheny College, the answer it depends seems to be an appropriate response to determine which algae will arise in specific rivers.
\\
\\
5. A commentary on the challenges that you faced and the way(s) that you overcame them.
\\
During the completing of this laboratory assignment, there was one specific challenge that I faced. The biggest challenge that I had faced was understand and using the command of the R programming language for statistical computation. It had took me a long amount of time to be able to understand and use the commands that the R programming language provides. To overcome this issue, I decided to use the IDE RStudio to get myself better accustomed using R. Another challenge that I faced during this assignment was being able to ignore the NA when calculating the mean for the chlorophyll. To overcome this issue I was researching through the documentation of R used the command na.rm = TRUE to be able to ignore all the NA in the data when calculating the mean of chlorophyll.
\end{document}